%!TEX root = main.tex
\newpage
\section{State of the Art}

Nowadays, people use many services based in the cloud and many companies choose to use them too. By doing that, companies reduce the costs of IT infrastructure and not even need to buy ``physical storage'', neither care where the data is. The cloud service provides that the data is secure.
However, like any system, the cloud has problems such any other computer system, software and hardware faults. The resilience of the cloud is very important too.
\iftoggle{long}{\orange{Estas frases exemplificam um argumento pouco claro. Existe um “but” a unir as locuções, mas no entanto não há qualquer relação entre elas. Como resultado, é pouco claro o que se pretende dizer.}}
The increased use of cloud is related to a low usage of many dedicated servers, lower voltage levels, reduction of noise margins and increasing clock rates. The cloud provider offer resources ready to deliver \cite{wolter2012resilience}.

There are many studies showing that the software faults\cite{avizzienisbasic} are the main cause of computer failures. But, the number of faults that can be emulated is directly related to the technique used.

\begin{table}[h]
\centering
\begin{tabular}{c|c|c}
         & \textbf{Software} & \textbf{Hardware} \\ \hline
\textbf{Hardware} &          & HWIFI    \\ \hline
\textbf{Software} & SWIFI    & SWIFI
\end{tabular}
\caption{\small \sl Fault injection techniques and emulation environment.}
\label{tab:swifi_hwifi}
\end{table}

In the Table \ref{tab:swifi_hwifi}, it is possible to view that can be used SWIFI technique to take software to emulate software and hardware faults, and can be user HWIFI technique to emulate hardware faults throught hardware.

\begin{itemize}
	\item \textbf{\ac{swifi}} - the goal of this technique is to emulate errors at software level that happen during the execution environment, in hardware or software; Examples: Data corruption in registers, memory or hard drive; Communication problems in network or NoC; Software faults in binary code, in object files or in source code.

	\item \textbf{\ac{hwifi}} - this technique is related to the fault injections in the final system hardware. Examples: \ac{emp}, radiation.

\end{itemize}

\Ac{swifi} are an attractive technique because won't require additional hardware (increase the cost of test). The targets of this technique are the applications and the operating systems, but, this technique don't have only advantages, can't inject faults in inaccessible areas of software and may disrupt or change the workload of the testing software. This technique can be used at:

\begin{itemize}
	\item \textbf{Compilation time (object code level)} 		- Modify the structure of the program before the creation of executable file;
	 % and can emulate Hardware faults, software faults;
	\item \textbf{Execution environment (binary code level)} 	- Changing the binary code activated by a timeout, an exception or a trap. At this level, less than seventy percent of the software faults can be emulated \cite{madeira2000emulation}.
	\item \textbf{Before compile time (source code level)} 		- Derivate the source code by removing, replacing or inserting some simple code before the compilation of program;
\end{itemize}

\iftoggle{long}{\red{With this work, I pretend to inject software faults and analyze how the system reacts to them.}}

%\iftoggle{long}{In Table \ref{tab:representative_faults}, are specified the most representative fault types, they represent a total of 67\% of all faults collected \cite{duraes2005thesis}.}


\iftoggle{long}{\orange{O estilo continua com frases soltas, e pouco articuladas.
Não sei se a maior parte das avarias são causadas por falhas de software. Há estudos que indicam que erros de operador são igualmente muito comuns. Por que razão é que ``less than 70\% of the software faults can be emulated !?'' Podem ser todas emuladas. Algumas técnicas, p.ex. injeção no binário executável, é que podem não permitir injetar algumas falhas. No entanto, ao nível do código fonte, 100\% das software faults são emuláveis.}}

% \begin{table}[!ht]
% \begin{tabular}{c}
% \includegraphics[width=1\textwidth]{representative_faults.jpg}
% \end{tabular}
% \caption{\small \sl \red{Fault coverage of the representative fault types.} \cite{duraes2005thesis} \label{tab:representative_faults}}
% \end{table}


%In this work
%deliberate how


% \marginnote{especificar as abreviaturas...}[0cm]
% \\textbf{acs{cots}} A \acl{cots}
% \acl{g-swfit}
% \acl{odc}

I had the opportunity to access to the application (executable only)
%not the source)
of Robert Natella, named SAFE, that injects software faults, as I also intended to do (I will describe it in next section).

\clearpage
\subsection{Software Implemented Fault Injection of Software Faults}
% \iftoggle{long}{\red{In the next subsections, I will describe some fault injectors that have been previously done.}\\}

Above, I will describe some tools that use \ac{swifi} technique and made some improvements in the area of this research.

% \iftoggle{long}{
%  	\subsubsection{Doctor}
% 	\hypertarget{Doctor}{}
%  	\bookmark[level=\thesubsubsection,dest=Doctor]{\thesubsubsection \ Doctor}
% }

% \iftoggle{long}{
%  	\subsubsection{Ferrari}
% 	\hypertarget{Ferrari}{}
%  	\bookmark[level=\thesubsubsection,dest=Ferrari]{\thesubsubsection \ Ferrari}
% }

% \iftoggle{long}{
%  	\subsubsection{Ftape}
% 	\hypertarget{Ftape}{}
%  	\bookmark[level=\thesubsubsection,dest=Ftape]{\thesubsubsection \ Ftape}
% }

\subsubsection{JACA Tool}
\hypertarget{JACA Tool}{}
\bookmark[level=\thesubsubsection,dest=JACA Tool]{\thesubsubsection \ JACA Tool}

JACA\cite{regina2003jaca} is a tool that has been made to validate Java applications. It injects high-level software faults and is based on computational reflection to inject interface faults in Java applications
\cite{martins2002jaca}. \\

\subsubsection{J-SWFIT}
\hypertarget{J-SWFIT}{}
\bookmark[level=\thesubsubsection,dest=J-SWFIT]{\thesubsubsection \ J-SWFIT}

Java Software Fault Injection Tool\cite{sanches2011j} is a tool that doesn't need the source code to perform the injection, the mutation of the code is performed directly at byte-code level.\\

\subsubsection{SAFE by Robert Natella}
\hypertarget{SAFE by Robert Natella}{}
\bookmark[level=\thesubsubsection,dest=SAFE by Robert Natella]{\thesubsubsection \ SAFE by Robert Natella}

Safe is an application to inject realistic software faults in programs coded in C and C++.
This tool uses MCPP as parser, to get the tree of code. The decision of using MCPP instead of GCC parser was a workaround for some of the shortcomings of the GCC's C preprocessor.

After that, write some files, variations of original files (code with simple mutations) with operators applied.
Robert Natella implemented thirteen operators in SAFE, same as João Durães\cite{duraes2006emulation}, but with the difference that Robert implemented at source code level, and João at binary level.\\


% \iftoggle{long}{
%  	\subsubsection{Xception}
% 	\hypertarget{Xception}{}
%  	\bookmark[level=\thesubsubsection,dest=Xception]{\thesubsubsection \ Xception}
% }

\clearpage
\subsection{ODC Model}
\acl{odc}\cite{bridge1998orthogonal} Model is a framework developed by IBM\cite{chillarege2004orthogonal}, created to improve the level of technology available to assist the decisions of a software engineer, via measurement and analysis.
ODC can be used to classify and analyze defects during software development.

%\cite{lyu1996handbook}
For that, this model has eight categories:

\begin{itemize}
	\item \textbf{Function} - This defect affects significant capability, end-user features, product \acl{api}, interface with hardware architecture, or global structure(s). It would require a formal design change.
	\item \textbf{Assignment} - Typically an assignment defect indicates an initialization of control blocks or a data structure.
	\item \textbf{Interface} - Problems in the interaction with other components, modules, device drivers, call statements, control blocks, or parameter lists.
	\item \textbf{Checking} - Based on the program logic that is checked and failed to validate data and values before the usage, loop conditions, etc.
	\item \textbf{Timing/serialization} - Errors that happen in shared and real-time resources.
	\item \textbf{Build/package/merge} - Errors that occur in the integration of library systems, management of changes, or in version control.
	\item \textbf{Documentation} - Errors in the documentation, that can be propagated to publications and maintenance notes.
	\item \textbf{Algorithm} - Problems that can be fixed by re-implementing an algorithm or local data structure, include efficiency or correctness that affect the task.
\end{itemize}


\newpage
\section{Research objectives and approach method}

In this section are discussed the main aspects in study.

\subsection{Cloud Computing}



To understand a little more what the Cloud Computing means:\\

\textit{``Cloud computing is a model for enabling ubiquitous, convenient, on-demand network access to a shared pool of configurable computing resources (e.g., networks, servers, storage, applications, and services) that can be rapidly provisioned and released with minimal management effort or service provider interaction.''}\cite{mell2011nist}.\\

Cloud Computing is a new way to delivery IT services on-demand (utility-oriented and Internet-centric). This services include all the computational power, from hardware infrastructure as a set of virtual machines to software services as development platforms and distributed applications.

\begin{figure}[!ht]
\begin{center}
\includegraphics[width=0.8\textwidth]{cloudcomputing.jpg}
\caption{\small \sl Cloud computing overview.\label{fig:cloudcomputing}}
\end{center}
\end{figure}

Below, I will describe it in relation to characteristics, deployment models and service models\cite{schouten2013ibm}.

The characteristics of Cloud Computing are:
\begin{itemize}
	\item \textbf{On demand self-service} 	- The users can request and manage their cloud computing resources without requiring human interaction, through a web-based self-service portal.
	\item \textbf{Broad network access 	}	- Provide access over the network and using standard way through by several clients (e.g., mobile phones, tablets, laptops and workstations).
	\item \textbf{Resource pooling 		}	- The computer resources are pooled to serve multiple customers through the safe separation of the resources at logical level.
	\item \textbf{Rapid elasticity 		}	- Capability of resources to be elastically provisioned and released. Making sure that the application will have exactly the capacity that it needs at any point of time.
	\item \textbf{Measured service 		}	- The service is monitored, measured, and reported transparently based on the usage. The clients pay in accordance with the service spent.
\end{itemize}

Four models of deployment:
\begin{itemize}
	%and treat effectively a lot of requests.}
	\item \textbf{Private Cloud}   - It is a single-tenant cloud solution utilizing client hardware and software, is located inside the client firewall or even data center. The sensitive information is maintained inside of organization. It has the disadvantage of not having ability to scale on demand.

	\item \textbf{Community Cloud} - It is shared by organizations with similar interests, supported by a specific community, sharing the same mission, security requirements, etc.

	\item \textbf{Public Cloud}    - It is available to the general public or to a group of a big company. It is a multi-tenant cloud solution owned by cloud service provider, that delivers shared hardware and software to clients private network (mostly the Internet) and data centers.

	\item \textbf{Hybrid Cloud}    - Composed by two or more services (private, community or public), together by standard technologies or proprietary that allows portability. Takes advantages from the best of private and public. Example: A client can implement a private cloud for applications with sensitive data and a public cloud for other data, non-sensitive.
\end{itemize}

Four levels of Cloud Computing Service Models:

\begin{itemize}
	\item \textbf{\acl{iaas}} - As the name suggests, provides a computing infrastructure, such as virtual machines, firewalls, load balancers, IP addresses, virtual local area networks and others. Examples: Amazon EC2, Windows Azure.

	\item \textbf{\acl{paas}} - Provides a computing platform, normally includes operating system, programming language execution environment, database, web server and others. Examples: AWS Elastic Beanstalk, Windows Azure, Heroku.

	\item \textbf{\acl{saas}} - Provides access to application softwares often referred as \textit{on-demand self-service} software. Use it without install, setup, and run the application. Service provider does all those things for you. Examples: Google Apps, Microsoft Office 365.

	\item \textbf{\acl{bpaas}} - This model provides an entire horizontal or vertical business process and builds on top of any of services previously described.

\end{itemize}

\begin{figure}[!ht]
\begin{center}
\includegraphics[width=1\textwidth]{cloud-computing-service-models2.png}
\caption{\small \sl Cloud computing service models.\label{fig:cloudcomputingservicemodels}}
\end{center}
\end{figure}

Nevertheless, such as any computer system, cloud computing isn't free of external disturbances\cite{wolter2012resilience}, the most important are:
\begin{itemize}
 	\item \textbf{Security attacks - } any try to gain unauthorized access;
 	\item \textbf{Accidents - } an unplanned event;
 	\item \textbf{Power surges - } an interruption of the flow of electricity;
 	% \item \textbf{Workload faults - } cenas;
 	\item \textbf{Malfunction - } bugs cause function wrongly, or not function at all;
 	\item \textbf{Worms - } malware computer program;
 	\item \textbf{\acl{ddos} attacks - } a try to make an network resource unavailable.
 \end{itemize}

\clearpage
\subsection{Tools}
 % - GCC Parser, Bison and Eclipse CDT}

In the beginning of planning the basic software without any user interface, it was necessary to research the best applications, as the best way for using them to obtain panned results (fault injector).
For that, I thought that I could use the same tools that I have used in Compilers course, Lex and Yacc.\\

%\red{For parsing the code, analyze and modify it,}



\iftoggle{long}{
 	\subsubsection{Bison/Yacc}
	\hypertarget{Bison/Yacc}{}
 	\bookmark[level=\thesubsubsection,dest=Bison/Yacc]{\thesubsubsection \ Bison/Yacc}
}

\subsubsection{Eclipse CDT}
\hypertarget{Eclipse CDT}{}
\bookmark[level=\thesubsubsection,dest=Eclipse CDT]{\thesubsubsection \ Eclipse CDT}

Eclipse CDT, as the name suggests, is a plugin for Eclipse that provides a fully functional C and C++ Integrated Development Environment.
Some of the features included in this plugin that are interesting for this project are:
\begin{itemize}
	\item Source navigation;
	\item Code editor with syntax highlighting;
	\item Source code refactoring and code generation.
\end{itemize}

It's possible to use this plugin in standalone mode, importing .jar files to the project.
Using it, I can code Fault Injector in Java, making the software more maintainable and easy to use, write, compile and debug.
\\

\subsubsection{GCC Parser}
\hypertarget{GCC Parser}{}
\bookmark[level=\thesubsubsection,dest=GCC Parser]{\thesubsubsection \ GCC Parser}

Nowadays, GCC use a hand-written parser to improve syntactic error diagnostics, giving people meaningful messages on syntax errors.\\

\iftoggle{long}{
 	\subsubsection{MCPP}
	\hypertarget{MCPP}{}
 	\bookmark[level=\thesubsubsection,dest=MCPP]{\thesubsubsection \ MCPP}
}

% Finally, I decided to use
In the end, I selected Eclipse CDT Plugin as standalone (only import libraries to project), because of my abilities in programming in Java Language, the maintainability of software, the low learning level than the developers need to modify it.\\

\newpage
% \section{Current work and preliminary results}
\section{Fault Injector Development}

The Fault Injector currently in development is coded in Java using Eclipse CDT, and it will have thirteen operators (can be seen in Table \ref{tab:faultEmulationOperators})\cite{duraes2005thesis}. However, the fault injector can have two schema's of trigger the faults: spatial and temporal. In the temporal way, the insertion of the fault is given by the time associate with the execution in system. Whereas, in the spatial way, the fault is injected when reaches the specified zone where the particular operator can be applied.

\begin{table}[!ht]
\begin{tabular}{|l|p{12cm}|}
\hline
\textbf{Fault Type}		& \multicolumn{1}{c|}{\textbf{Description}}		\\ \hline \hline
MFC        				& \Acl{mfc}  									\\ \hline
MIA        				& \Acl{mia}  									\\ \hline
MIEB       				& \Acl{mieb} 									\\ \hline
MIFS       				& \Acl{mifs} 									\\ \hline
MLAC       				& \Acl{mlac} 									\\ \hline
MLOC       				& \Acl{mloc} 									\\ \hline
MLPA       				& \Acl{mlpa} 									\\ \hline
MVAE       				& \Acl{mvae} 									\\ \hline
MVAV       				& \Acl{mvav} 									\\ \hline
MVIV       				& \Acl{mviv} 									\\ \hline
WAEP       				& \Acl{waep} 									\\ \hline
WPFV       				& \Acl{wpfv} 									\\ \hline
WVAV       				& \Acl{wvav} 									\\ \hline
\end{tabular}
\caption{\small \sl Fault emulation operators.\label{tab:faultEmulationOperators}}
\end{table}

In the beginning of this project,
\iftoggle{long}{\orange{Convém explicar e contextualizar tudo isto. Nas notas ao longo do semestre creio que há muitos argumentos que fomos colecionando.}}

\subsection{Generate derivations}

I chose to use a set of the most representative faults, previously specified by João Durães\cite{duraes2006emulation} according to his data-field results, specified individually further down:

	\subsubsection{\textbf{\acs{mfc}}} - \Acl{mfc}
	\hypertarget{mfc}{}
	\bookmark[level=\thesubsubsection,dest=mfc]{\thesubsubsection \ \acs{mfc}}

	The emulation of this operator is based in the remotion of a function call in a context where the returned value is not used. But, to do the remotion, the constraints below need to be validated.
	\begin{itemize}
		\item \textbf{\acs{c01}} - \Acl{c01};
		\item \textbf{\acs{c02}} - \Acl{c02}.
	\end{itemize}

	\subsubsection{\textbf{\acs{mia}}} - \Acl{mia} - \green{\textbf{Implemented}}
	\hypertarget{mia}{}
	\bookmark[level=\thesubsubsection,dest=mia]{\thesubsubsection \ \acs{mia}}

	This operator simulate a missing \textit{if} condition surrounding a set of statements. This causes that the statements are always executed and not only when the condition of \textit{if} is true.
	\begin{itemize}
		\item \textbf{\acs{c08}} - \Acl{c08};
		\item \textbf{\acs{c09}} - \Acl{c09}.
	\end{itemize}

	\subsubsection{\textbf{\acs{mieb}}} - \Acl{mieb} - \green{\textbf{Implemented}}
	\hypertarget{mieb}{}
	\bookmark[level=\thesubsubsection,dest=mieb]{\thesubsubsection \ \acs{mieb}}

	This operator generates derivations of the source code of applications by removing the if construct plus statements plus else before statements. To apply this operator I need to verify the constraint above:
	\begin{itemize}
		\item \textbf{\acs{c08n}} - \Acl{c08n}.
	\end{itemize}

	This constraint does not exists in João Durães specification, but as this operator cannot be applied in all situations, I specify and implement it.

	\subsubsection{\textbf{\acs{mifs}}} - \Acl{mifs} - \green{\textbf{Implemented}}
	\hypertarget{mifs}{}
 	\bookmark[level=\thesubsubsection,dest=mifs]{\thesubsubsection \ \acs{mifs}}

	The application of this operator changes the source code with the remotion of one \textit{if} construct and the statements surrounded by it.
	But, to do that, I need to verify the constraints above:
	\begin{itemize}
		\item \textbf{\acs{c02}} - \Acl{c02};
		\item \textbf{\acs{c08}} - \Acl{c08};
		\item \textbf{\acs{c09}} - \Acl{c09}.
	\end{itemize}

	\subsubsection{\textbf{\acs{mlac}}} - \Acl{mlac} - \green{\textbf{Implemented}}
	\hypertarget{mlac}{}
	\bookmark[level=\thesubsubsection,dest=mlac]{\thesubsubsection \ \acs{mlac}}

	This operator emulates the remotion of part of a logical expression used in a branch condition. To apply this operator, the code must have at least two branch conditions linked together with the logical operator AND. With an AND operator, if one of the sub-expressions is \textit{false} all the expression will be \textit{false} and the condition will fail.
	\begin{itemize}
		\item \textbf{\acs{c12}} - \Acl{c12}.
	\end{itemize}

	\subsubsection{\textbf{\acs{mloc}}} - \Acl{mloc} - \green{\textbf{Implemented}}
	\hypertarget{mloc}{}
	\bookmark[level=\thesubsubsection,dest=mloc]{\thesubsubsection \ \acs{mloc}}

	This operator emulate the remotion of part of a logical expression used in a branch condition. To apply this operator, the code must have at least two branch conditions linked together with the logical operator OR. It is only necessary that one of the sub-expressions be true to the entire expression are evaluated as true. This operator have only one constraint:

	\begin{itemize}
		\item \textbf{\acs{c12}} - \Acl{c12}.
	\end{itemize}

	\subsubsection{\textbf{\acs{mlpa}}} - \Acl{mlpa}
	\hypertarget{mlpa}{}
	\bookmark[level=\thesubsubsection,dest=mlpa]{\thesubsubsection \ \acs{mlpa}}

	As the name suggests, this operator emulate the omission of a small and localized part of the algorithm.

	\begin{itemize}
		\item \textbf{\acs{c02}} - \Acl{c02};
		\item \textbf{\acs{c10}} - \Acl{c10}.
	\end{itemize}

	The constraint \textbf{\ac{c02}} guarantees that don't be removed all the statements in a block, because this would not correspond to a realistic fault. This type of faults never involved the remotion of \textit{if} or \textit{if-else} and loop constructs (the omitted statements were always function calls and assignments) guaranteed by constraint \textbf{\ac{c10}}.

	\subsubsection{\textbf{\acs{mvae}}} - \Acl{mvae}
	\hypertarget{mvae}{}
	\bookmark[level=\thesubsubsection,dest=mvae]{\thesubsubsection \ \acs{mvae}}

	This operator reproduces the omission of a given local variable with an expression. But not when is the first assignment to a variable, an initialization, guaranteed by the constraint \textbf{\acs{c07}}.

	\begin{itemize}
		\item \textbf{\acs{c02}} - \Acl{c02};
		\item \textbf{\acs{c03}} - \Acl{c03};
		\item \textbf{\acs{c06}} - \Acl{c06};
		\item \textbf{\acs{c07}} - \Acl{c07}.
	\end{itemize}

	\subsubsection{\textbf{\acs{mvav}}} - \Acl{mvav}
	\hypertarget{mvav}{}
	\bookmark[level=\thesubsubsection,dest=mvav]{\thesubsubsection \ \acs{mvav}}

	Operator \textbf{\ac{mvav}} is similar to operator \textbf{\ac{mvae}}, with the difference that it emulate the remotion of the assignment of a given local variable with a constant value instead of a expression. The constraints related with this operator are the same of \textbf{\ac{mvae}}:

	\begin{itemize}
		\item \textbf{\acs{c02}} - \Acl{c02};
		\item \textbf{\acs{c03}} - \Acl{c03};
		\item \textbf{\acs{c06}} - \Acl{c06};
		\item \textbf{\acs{c07}} - \Acl{c07}.
	\end{itemize}

	\subsubsection{\textbf{\acs{mviv}}} - \Acl{mviv}
	\hypertarget{mviv}{}
	\bookmark[level=\thesubsubsection,dest=mviv]{\thesubsubsection \ \acs{mviv}}

	As the name suggests, this operator represents the remotion of a given local variable initialization with a constant value. The fact that this operator only search for variable initialization induce that only the first occurrence of an assignment to a particular variable are readable to apply this type of fault, this is guaranteed by the constraint \textbf{\acs{c04}}. The constraint \textbf{\acs{c05}} verify if the assignment don't occurs inside a loop, because one assignment of this type occurs several times. But this operator has other associated constraints:

	\begin{itemize}
		\item \textbf{\acs{c02}} - \Acl{c02};
		\item \textbf{\acs{c03}} - \Acl{c03};
		\item \textbf{\acs{c04}} - \Acl{c04};
		\item \textbf{\acs{c05}} - \Acl{c05};
		\item \textbf{\acs{c06}} - \Acl{c06}.
	\end{itemize}


%	\hypertarget{wlec}{}
%	\subsubsection{\textbf{\acs{wlec}}} - \Acl{wlec}
%	\hypertarget{wall}{}
%	\subsubsection{\textbf{\acs{wall}}} - \Acl{wall}


	\subsubsection{\textbf{\acs{waep}}} - \Acl{waep}
	\hypertarget{waep}{}
	\bookmark[level=\thesubsubsection,dest=waep]{\thesubsubsection \ \acs{waep}}

	This operator represents the

%	\hypertarget{wsut}{}
%	\subsubsection{\textbf{\acs{wsut}}} - \Acl{wsut}

	\subsubsection{\textbf{\acs{wpfv}}} - \Acl{wpfv}
	\hypertarget{wpfv}{}
	\bookmark[level=\thesubsubsection,dest=wpfv]{\thesubsubsection \ \acs{wpfv}}

	\begin{itemize}
		\item \textbf{\acs{c03}} - \Acl{c03};
		\item \textbf{\acs{c11}} - \Acl{c11}.
	\end{itemize}

	\subsubsection{\textbf{\acs{wvav}}} - \Acl{wvav}
	\hypertarget{wvav}{}
	\bookmark[level=\thesubsubsection,dest=wvav]{\thesubsubsection \ \acs{wvav}}

	As the name suggests, this operator simulate an assignment of a wrong value to a variable. To do that, the operator

	\begin{itemize}
		\item \textbf{\acs{c03}} - \Acl{c03};
		\item \textbf{\acs{c04}} - \Acl{c04};
		\item \textbf{\acs{c06}} - \Acl{c06}.
	\end{itemize}

The operators above will be applied to source code of applications and will generate modified files.

%\subsubsection{Fault Types - Extraneous}
%\begin{itemize}
%	\item \textbf{EVAV} - variable assignment using another variable
%\end{itemize}
\clearpage
\subsection{Constraints}

As was discussed in the specification of the operators, the same can be applied in all the situations and need to be applied in accordance with the field-data study, that have been done by João Durães.

\begin{table}[!ht]
\centering
\begin{tabular}{|c|p{12cm}|}
\hline
\textbf{Constraints}            & \multicolumn{1}{c|}{\textbf{Description}}                                     \\ \hline \hline
\textbf{C01}         & \Acl{c01} \\ \hline
\textbf{C02}         & \Acl{c02} \\ \hline
\textbf{C03}         & \Acl{c03} \\ \hline
\textbf{C04}         & \Acl{c04} \\ \hline
\textbf{C05}         & \Acl{c05} \\ \hline
\textbf{C06}         & \Acl{c06} \\ \hline
\textbf{C07}         & \Acl{c07} \\ \hline
\textbf{C08}         & \Acl{c08} \\ \hline
\textbf{C09}         & \Acl{c09} \\ \hline
\textbf{C10}         & \Acl{c10} \\ \hline
\textbf{C11}         & \Acl{c11} \\ \hline
\end{tabular}
\caption{\small \sl Fault emulation contraints defined by João Durães.\label{tab:faultEmulationConstraintsDuraes}}
\end{table}

The constraint \textbf{C07} is similar to constraint \textbf{C04}, one is the negation of another. This happen too with the constraint \textbf{C08} and \textbf{C08n}.
Constraint \textbf{c10} are the same as constraint \textbf{C09}, but with one additional restriction, the statements need to be contiguous and need to belong to the same code block.

When was implementing the operators \textbf{\ac{mieb}} and \textbf{\ac{mloc}}, it was necessary to define the constraints \textbf{\ac{c08n}} and \textbf{\ac{c12}}. The constraint \textbf{\ac{c08n}} was created because of the operator \textbf{\ac{mieb}} cannot be applied to an \textit{if} without an \textit{else} construct and the constraint \textbf{\ac{c12}} was created because the operator \textbf{\ac{mloc}} can't be emulated in a branch with only one condition.

\begin{table}[!ht]
\centering
\begin{tabular}{|c|p{12cm}|}
\hline
\textbf{Constraints}            & \multicolumn{1}{c|}{\textbf{Description}}                                     \\ \hline \hline
\textbf{C08n}         & \Acl{c08n} \\ \hline
\textbf{C12}         & \Acl{c12} \\ \hline
\end{tabular}
\caption{\small \sl Other constraints.\label{tab:otherConstraints}}
\end{table}

This constraints can be modified during the implementation of the other operators that aren't yet implemented.

\clearpage
\section{Work plan and implications}

In the figure \ref{fig:decisions}, can be seen an overview of the main decisions that I did during the first semester of this dissertation. Since the beginning of this project, it was expected to create a fault injector.

\begin{figure}[!ht]
\begin{center}
\includegraphics[width=0.8\textwidth]{decisions.png}
\caption{\small \sl Decision tree.\label{fig:decisions}}
\end{center}
\end{figure}

\begin{figure}[!ht]
\begin{center}
\includegraphics[width=1\textwidth]{mockup.png}
\caption{\small \sl Overview of the injection tool.\label{fig:mockup}}
\end{center}
\end{figure}

\iftoggle{long}{\red{Select five to ten programs to be tested.}}


\iftoggle{long}{\red{Justificar a utilização de patchs}}


Built three separated modules:

\begin{itemize}
	\item Generate the derivations of main code of selected programs;
	\item Verify and analyze the effect of produced faults;
	\item Compile the programs with injected faults, by using make file.
\end{itemize}


\iftoggle{long}{\red{Problems with the rewriting of tree}}


\iftoggle{long}{\red{Reflection}}



\iftoggle{long}{\red{But I was forced to take decisions after that, for example, after creating the tree of code, I can go through the tree in the recursive way or using \textit{Visitor Pattern}.}}







\subsection{Analyze the effects}

\iftoggle{long}{\red{The fault injected results are equal to the real software faults?}}

After the compilation and execution of the programs, the results need to be evaluated. To measure that, I will use the \textit{Koopman's CRASH Scale}\cite{koopman1997comparing}:

\begin{itemize}
	\item \textbf{C}atastrophic - Operating System crashed or multiple tasks affected;
	\item \textbf{R}estart - Task or process hangs, requiring restart;
	\item \textbf{A}bort - Task or process aborts abnormally (i.e. ``code dump'' or ``segmentation violation'');
	\item \textbf{S}ilent - Test Process exits without an error code returned when one should exist;
	\item \textbf{H}indering - Test Process exits with an error code not relevant to the situation or incorrect error code returned;
	\item \textbf{P}ass - The module exits properly, possibly with an appropriate error code.
\end{itemize}

This \textit{CRASH Scale} is one way to show results of the effect of faults on an end-use system, mainly from the operating system perspective.

\iftoggle{long}{\orange{Esta parte é particularmente importante: a classificação dos efeitos. Creio que há mais a dizer, pois fará parte do plano futuro de desenvolvimento classificar o que cada programa defeituoso faz.}}


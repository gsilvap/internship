%!TEX root = main.tex
\newpage
% \chapter{Introduction}
\section{Introduction}

\subsection{Contextualization}
The present dissertation describe the work developed in scope of MSc in Informatics Engineering. It is focused in ``Evaluate the robusteness of Cloud'' and this is one subject very important of nowadays, because of the increase usage of clouding services.
This services are characterized by the placement of data and software on remote infrastructure. Despite of the numerous benefits, the reliability of these platforms has not kept the needs, and users trust their applications to systems outside of personal control. 

In this context, naturally arises the problem of confidence in the entity that manages the platform where applications have been executed. Any organization that put an application in the cloud (for example, Microsoft Azure or Amazon EC2) will have to accept the assurances given by the service provider.

This internship deals with the challenge of assessing the robustness of cloud platforms. The computing service provider uses virtualization to manage and allocate computing power to meet actual needs of the application. Although, there are solid virtualization platforms, fault tolerance is still a research problem.


% \citewwwbib{ref02}
% \cite{ref02}

\subsection{The project}

This project is based essentially in inject software faults, in hardware and software.

\subsection{Objectives}

The main objective of this work is to build a tool to inject software faults in code of some programs before the compilation.

But this main objetive is divided in some other goals:

\begin{itemize}
	\item Generate de derivations of main code of selected programs;
	\item Verify and analyse the effect of producced faults;
	\item Compile the programs with injected faults, by using make file.
\end{itemize}



\subsection{Document Structure}

This document is ...

In the second section ...

In the third section ...

Finally, in last section ...
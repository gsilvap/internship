%!TEX root = main.tex
\clearpage
% \chapter{Introduction}
\section{Introduction}

This internship deals with the challenge of assessing the robustness of cloud platforms. The computing service provider uses virtualization to manage and allocate computing power to meet present needs of the application.
Faults will be injected in software running in the cloud and collected results will be evaluated.
% Will be injected faults in software in the cloud and collected results to be evaluated.
% \iftoggle{long}{\red{In the next subsections will be introduced the context and the scope of this project.}
% \orange{1,2 ou 3 paragrafos descritivos do problema}}

\subsection{Contextualization}


The present dissertation describes the work developed in the scope of Master degree in Informatics Engineering. It is focused on evaluating the robustness of the cloud computing (usually simply called ``the cloud'' or ``cloud''), which is a very important issue nowadays, mostly because of its increasing usage.
It is characterized by the placement of data and software and services on remote infrastructures. \red{Despite its numerous benefits, the reliability of these platforms has not kept the needs, and users trust on their applications to systems outside of personal control.}

In this context, the problem of the existence of bugs in software of the entity managing the platform where applications are  executed arises naturally. There are many reasons for the existence of these software bugs, the most important being:

\begin{itemize}
	\item Miscommunication;
	\item Software complexity;
	\item Programming errors;
	\item Changing requirements;
	\item Time pressures;
	\item Egotistical or overconfident people;
	\item Poorly documented code;
	\item Software development tools;
	\item Obsolete automation scripts;
	\item Lack of skilled testers.
\end{itemize}

Therefore, the bugs will continue to exist and will always exist. Because of this, the test of the ability of a critical system to deal with existing bugs is critical. This is the main reason for the existence of this dissertation.

Any organization that puts an application in the cloud (for instance in Microsoft Azure or Amazon EC2) should accept the assurances given by the service provider.

% \iftoggle{long}{\orange{Logo aqui deve ficar claro que os bugs existem, vão sempre existir, e que testar a capacidade de qualquer sistema crítico para lidar com bugs existentes é fundamental. Daí todo o trabalho. Não é claro por que razão (ou razões) isto é específico para a Cloud. Não é igual fazer-se para a Cloud ou para programas stand-alone clássicos?}}


Although there are solid virtualization platforms, fault tolerance is still a problem in research. The system's ability to recover from the failures existence, named resilience, is a critical factor in the cloud.

% \iftoggle{long}{\red{resilience}}


% \subsection{The project}

% This project is based mainly in inject software faults. It was decided since there are already other people involved in the part of hardware faults.

\subsection{Objectives}

The main objective of this work is to evaluate the robustness of the cloud. To do that, I will design and implement a tool to inject software faults in the source code of some applications.

Nevertheless, this main objective is divided in some other goals:

% \iftoggle{long}{\orange{Esta parte está bem, mas o passo de “evaluate the robustness of the cloud” para estes 3 objetivos é demasiado grande. Deve explicar-se um pouco mais, partindo de um objetivo grande, e progressivamente estabelecer o que aqui se chama “goals”.}}

\begin{itemize}
	\item Implement the thirteen operators specified by João Durães;
	\item Use the fault injector to emulate faults in applications;
	\item Measure the time and the value obtained after running the application, in normal conditions;
	\item Inject a fault in an application, verify and analyze the effect. Measure the value and the time running the application with and without faults;
	\item Compare the time and value in a normal scenario and in a scenario with faults;
	\item Create a scenario with multiple virtual machines, verify and analyze the effect. Measure the value and the time with the application without faults and with faults;
	\item Compare all the results and obtain conclusions.

	% \item Generate derivations of main code of selected programs;
	% \item Verify and analyze the effect of produced faults;
	% \item Compile the programs with injected faults, by using make file.
\end{itemize}


\subsection{Document structure}

In this document are specified all the related subjects with the project.

The second section presents the state-of-the-art in the related areas with particular emphasis to the fault injectors of software faults.

The third section is an important section of this report, because of the research involved in the execution of this work. It was necessary to take some important decisions based in research results, knowledge and my own experience.

The fourth section describes the work that has been done in Fault Injector, and the work that should be done in the next semester.

The fifth section explains other modules that need to be executed in this project to observe and evaluate the results of the fault injector.

In the last section, I will do an overview analysis of my work, in general the operators and the constraints developed. I will also talk about the work to be done in the next semester.


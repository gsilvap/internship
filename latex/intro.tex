%!TEX root = main.tex
\newpage
% \chapter{Introduction}
\section{Introduction}

In the next subsections will be introduced the context and the scope of this project.

\subsection{Contextualization}
The present dissertation describes the work developed in scope of MSc in Informatics Engineering. It is focused in ``Evaluate the robusteness of Cloud'' and this is one subject very important nowadays, because of the increase usage of these services.
This services are characterized by the placement of data and software on remote infrastructure. Despite of the numerous benefits, the \red{reliability} of these platforms has not kept the needs, and users trust their applications to systems outside of personal control.

In this context, naturally arises the problem of \red{confidence} in the entity that manages the platform where applications have been executed. Any organization that put an application in the cloud (for example, Microsoft Azure or Amazon EC2) will have to accept the assurances given by the service provider.

This internship deals with the challenge of assessing the \red{robustness} of cloud platforms. The computing service provider uses virtualization to manage and allocate computing power to meet actual needs of the application. \red{Although, there are solid virtualization platforms, fault tolerance is still a research problem.}


% \citewwwbib{ref02}
% \cite{ref02}

\subsection{The project}

This project is based essentially in inject software faults. It was decided to inject software faults, since there are already other people involved in the part of hardware faults.

\subsection{Objectives}

The main objective of this work is to build a tool to inject software faults in code of some programs before the compilation.

But this main objective is divided in some other main goals:

\begin{itemize}
	\item Generate derivations of main code of selected programs;
	\item Verify and analyze the effect of produced faults;
	\item Compile the programs with injected faults, by using make file.
\end{itemize}


\subsection{Document Structure}

In this document are specified all the related subjects with the project.

The second section will present the state-of-the-art in the related areas with particular emphasis to Cloud Computing and Fault Injection.

Third section is an important section of this report, because of the research involved in the execution of this work. I had to make decisions based in research results, knowledge and my own experience.

Fourth section describe the work that have been done in Fault Injector, and the work that I have to do in the next semester.

Fifty section explain the other modules that need to be done in this project to can view and evaluate the results of the fault injector.

Finally, in last section I do an overview analyses to my work, in general the operators and the constraints developed. I speak also in the work to be done in the next semester.


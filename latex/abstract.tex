%!TEX root = main.tex
\clearpage
\pagenumbering{arabic}

% \addcontentsline{toc}{section}{Abstract}

\begin{abstract}
The Cloud Computing is a new paradigm that provides on demand self-service resources, broad network access, resource pooling, rapid elasticity and a measured service through four different models, public, private, hybrid and community.

The main objective of this paradigm is to allow users to get the most of the technology without having the knowledge and skills to ensure the proper functioning of all the technologies involved, allowing the users to focus on their core business, rather than be blocked due to the technological difficulties.

The fact of the virtualization be the fundamental technology that powers cloud computing, provide the reduction of IT cost while increase the efficiency, utilization and flexibility of their existing computer hardware.

However, the Cloud Computing isn't free of external disturbance like security attacks, power surges, workload faults, hardware and software faults. Due to this reason, the theme of my dissertation is ``Evaluate the robustness of the Cloud'' and it is based on the development of a fault injector software, to, as the name suggests, inject faults in software to testing it in the cloud later. After the testing, the collected results will be evaluated using the CRASH Scale.\\



% \iftoggle{long}{\orange{O abstract começa com um âmbito demasiado abrangente, e até um pouco fora do tema principal da tese. Tem de ser re-escrito para ficar mais focado.}}


% \iftoggle{long}{\red{Breve contextualixação antes de dizer objetivo!!!
% \red{This thesis/dissertation presents an ????}}


% Nowadays, the Information and Communication Technologies are responsible for 2-4\% of CO\textsuperscript{2} emissions, but in the next five or ten years these will increase to 10\% \cite{wolter2012resilience}. Because of this, the next challenge is to reduce the costs of ICT and its impact in the environment while the IC services keep growing.


% Cloud computing is a new paradigm that provides on-demand self-service resources (computing, network and storage). It also promises to reduce the costs of ICT, but isn't free of external disturbance like security attacks, power surges, workload faults and others.

% Therefore, the theme of my dissertation is ``Evaluate the robustness of the Cloud''. I will design and implement a fault injector for software coded in C to evaluate the capacity of the cloud to recover from faults.\\
% }


\textbf{Keywords:} Robustness, Cloud Computing, Faults, Errors, Failures, Vulnerabilities, Fault Injection, Fault Tolerance.

\end{abstract}

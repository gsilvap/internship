%!TEX root = main.tex
\subsection{Management}

% \iftoggle{long}{\red{In this section is described the planning of work developed in this dissertation.}}
\subsubsection{Meetings}
% \hypertarget{meetings}{}
% \bookmark[level=\thesubsubsection,dest=meetings]{\thesubsubsection \ Meetings}

About the meetings, the supervisor Raul Barbosa and I agreed that meeting once every week was the best option. Moreover, they happened, with one or another change of schedule to reconcile with the other activities from both. In addition, I attended some general meetings of the project. In them, we could discuss concepts and the direction of the project with colleagues and teachers, among them: Raul Barbosa (supervisor), Henrique Madeira (co-supervisor), João Durães and João André Ferro.

\subsubsection{Risks}
% \hypertarget{risks}{}
% \bookmark[level=\thesubsubsection,dest=risks]{\thesubsubsection \ Risks}

As any other projects, this project has risks too.
% Risks related to the planning and execution of this dissertation.
Some of the risks are related to equipment failure and data lost and to prevent that this happen I use \textit{GitHub} to backup the source developed to the project and this report. These backups are done in all days that I do some improvements to this project.

The particularity of this project is in investigation nowadays, bring other risk to this project, associated to the publication of similar research, to reduce this risk, I will check with regularity electronic publications, and if similar research was published, I will modify the project to assure that adds value and it's not just like any other.

Moreover, I can have personal issues interfering with the progress of this project or lose the interest, and to prevent this I have selected a motivating topic at the beginning and I talk to the supervisor always that I have doubts.

This risks, the preventative measures and the recovery measures can be seen, in other perspective, at Appendix \ref{App:B}.

\subsubsection{Planning and Tracking}
% \hypertarget{planning}{}
% \bookmark[level=\thesubsubsection,dest=planning]{\thesubsubsection \ Planning and Tracking}

In Appendix \ref{App:A}, is showed the Gantt diagram with the tasks that have been done during the first semester.
%  to first and to second semester.I'm not showing here the planned Gantt
As I postponed this dissertation for six months so, the scope and the context have changed. Now the two Gantt diagrams are incomparable.


%I have prioritized the tasks using the nomenclature in Table \ref{tab:classRequirements}:
%\begin{table}[!ht]
%\begin{tabular}{cc}
%\hline
%\textbf{Classification}                & \textbf{Mean}                                                                                                                                                           \\ \hline
%\multicolumn{1}{|c|}{\textit{Must}}    & \multicolumn{1}{c|}{\begin{tabular}[c]{@{}c@{}}Must be implement at project finish,\\  and his implementation is priority.\end{tabular}}                                \\ \hline
%\multicolumn{1}{|c|}{\textit{Nice}}    & \multicolumn{1}{c|}{\begin{tabular}[c]{@{}c@{}}May be part of the functionality implemented\\  at the end of project, and his implementation is optional.\end{tabular}} \\ \hline
%\multicolumn{1}{|c|}{\textit{Wishful}} & \multicolumn{1}{c|}{\begin{tabular}[c]{@{}c@{}}It's specified but his implementation is \\not expected until the end of project.\end{tabular}}                          \\ \hline
%\end{tabular}
%\caption {Classification of requirements} \label{tab:classRequirements}
%\end{table}

% \newpage
About the development of this project, I have used an \emph{Agile Life Cycle} based in an \emph{Incremental Model}. The use of this model is justified because of the small tasks that are being planned weekly.


\iftoggle{long}{\red{porque? ajuda? com que objectivo? foi uma boa opção? quais eram as alternativas? em que falhavam? porque nao foram escolhidas? }}


\iftoggle{long}{\red{What are the requirements of this project???}}


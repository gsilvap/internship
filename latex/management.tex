%!TEX root = main.tex
\newpage
\subsection{Management}

In this section is described the planning of work developed in dissertation.

\subsubsection{Meetings}
In relation a meetings, the supervisor Raul Barbosa and me agreed that meet weekly was the best option. And the meetings were going on, with one or another change of schedule to reconcile with the other activities of both. In addition, I went to several general meeting of the project. Where could discuss concepts and the direction of the project with colleagues and teachers, among them: Raul Barbosa (supervisor), Henrique Madeira (co-supervisor), João Durães and João André Ferro.

\subsubsection{Risks}

The main-risks of execution of this project are:

\begin{itemize}
	\item Equipment Failure
	\item Data lost
	\item Publication of similar research
	\item Personal issues interfere with progress
	\item Student loses interest
	\item Dispute between student and supervisor
	\item Supervisor takes excessive time to check final drafts
	\item Student wants to submit thesis without supervisor approval
\end{itemize}

% \red{Table of risks with cause, consequence and mitigation!!!}

The preventative measures and recovery measures can be seen at Appendix \ref{App:B}.

\subsubsection{Planning and Tracking}
In Appendix \ref{App:A}, is presented the gantt with the planning tasks to first and to second semester.

%I have prioritized the tasks using the nomenclature in Table \ref{tab:classRequirements}:
%\begin{table}[!ht]
%\begin{tabular}{cc}
%\hline
%\textbf{Classification}                & \textbf{Mean}                                                                                                                                                           \\ \hline
%\multicolumn{1}{|c|}{\textit{Must}}    & \multicolumn{1}{c|}{\begin{tabular}[c]{@{}c@{}}Must be implement at project finish,\\  and his implementation is priority.\end{tabular}}                                \\ \hline
%\multicolumn{1}{|c|}{\textit{Nice}}    & \multicolumn{1}{c|}{\begin{tabular}[c]{@{}c@{}}May be part of the functionality implemented\\  at the end of project, and his implementation is optional.\end{tabular}} \\ \hline
%\multicolumn{1}{|c|}{\textit{Wishful}} & \multicolumn{1}{c|}{\begin{tabular}[c]{@{}c@{}}It's specified but his implementation is \\not expected until the end of project.\end{tabular}}                          \\ \hline
%\end{tabular}
%\caption {Classification of requirements} \label{tab:classRequirements}
%\end{table}

\newpage
About the development of this project, I have used an \emph{Agile Life Cycle} based in a \emph{Incremental Model}.

\red{What is the requirements of this project???}

%!TEX root = main.tex
\subsection{Management}

% \iftoggle{long}{\red{In this section is described the planning of work developed in this dissertation.}}
\subsubsection{Meetings}
% \hypertarget{meetings}{}
% \bookmark[level=\thesubsubsection,dest=meetings]{\thesubsubsection \ Meetings}

Regarding meetings, supervisor Raul Barbosa and I agreed that meeting once every week was the best option. Moreover, they happened, with one or another change of schedule to reconcile with the other activities of both. In addition, I attended some general meetings of the project. In them, we could discuss concepts and the direction of the project with colleagues and teachers, among them: Raul Barbosa (supervisor), Henrique Madeira (co-supervisor), João Durães and João Fernandes.

\subsubsection{Risks}
% \hypertarget{risks}{}
% \bookmark[level=\thesubsubsection,dest=risks]{\thesubsubsection \ Risks}

As any other project, this one has risks too.
% Risks related to the planning and execution of this dissertation.
Some of the risks are related to equipment failure and data lost. To prevent that from happening I used \textit{GitHub} to backup all the sources developed for the project and this report. These backups are done on a daily basis and everytime I perform.

The particularity of the subject of this project is under intense investigation nowadays, bringing another kind of risks to this project, associated to the publication of similar research. To reduce this risk, I will regularly check with electronic publications, and if similar research was published, I will modify the project to assure that it adds additional value and not just like any other.

Moreover, I can have personal issues interfering with the progress of this project or lose the interest, and to prevent this I have selected a motivating topic from the beginning and every time I have doubts, I would talk to my supervisor.

These risks, the preventative measures and the recovery measures can be seen, in other perspective, at Appendix \ref{App:B}.

\subsubsection{Planning and tracking}
% \hypertarget{planning}{}
% \bookmark[level=\thesubsubsection,dest=planning]{\thesubsubsection \ Planning and Tracking}

In Appendix \ref{App:A}, is showed the Gantt diagram with the tasks that have been done during the first semester.
%  to first and to second semester.I'm not showing here the planned Gantt
As I postponed this dissertation for six months so, the scope and the context have changed. Now the two Gantt diagrams are incomparable.


%I have prioritized the tasks using the nomenclature in Table \ref{tab:classRequirements}:
%\begin{table}[!ht]
%\begin{tabular}{cc}
%\hline
%\textbf{Classification}                & \textbf{Mean}                                                                                                                                                           \\ \hline
%\multicolumn{1}{|c|}{\textit{Must}}    & \multicolumn{1}{c|}{\begin{tabular}[c]{@{}c@{}}Must be implement at project finish,\\  and his implementation is priority.\end{tabular}}                                \\ \hline
%\multicolumn{1}{|c|}{\textit{Nice}}    & \multicolumn{1}{c|}{\begin{tabular}[c]{@{}c@{}}May be part of the functionality implemented\\  at the end of project, and his implementation is optional.\end{tabular}} \\ \hline
%\multicolumn{1}{|c|}{\textit{Wishful}} & \multicolumn{1}{c|}{\begin{tabular}[c]{@{}c@{}}It's specified but his implementation is \\not expected until the end of project.\end{tabular}}                          \\ \hline
%\end{tabular}
%\caption {Classification of requirements} \label{tab:classRequirements}
%\end{table}

% \newpage
About the development of this project, I have used an \emph{Agile Life Cycle} based in an \emph{Incremental Model}.
% The use of this model is justified because of the small tasks that are being planned weekly.
New tasks are planned weekly, although it will always be a long term goal.
The use of this type of methodology is important because it easily plans the following tasks to overcome any difficulties that have appeared.


\iftoggle{long}{\red{porque? ajuda? com que objectivo? foi uma boa opção? quais eram as alternativas? em que falhavam? porque nao foram escolhidas? }}


\iftoggle{long}{\red{What are the requirements of this project???}}

